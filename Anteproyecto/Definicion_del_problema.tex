%% ---------------------------------------------------------------------------
%% Definicion_del_problema.tex
%%
%% Definicion_del_problema
%%
%% rojiark
%% ---------------------------------------------------------------------------

\chapter{Definición del problema}
\label{ch:Definicion_del_problema}


\section{Generalidades}

Como generalidad del entorno de los sistemas embebidos se encuentra la dificultad para 
los desarrolladores de trasladares de un paradigma de programación secuencial a uno que 
implique la utilización de múltiples unidades de procesamiento en forma paralela, de las nuevas
generaciones de plataformas multinúcleo, reutilizando para ello código ya existente (\textit{Legacy Code}).\\

En términos generales existen herramientas para la plataforma KeyStone I, sin embargo, éstas
no se encuentran optimizadas para un mejor consumo de los recursos y de la energía, por lo 
que la optimización de las mismas es un aspecto a considerar.\\

De forma más específica, para la plataforma KeyStone II no existe ninguna herramienta que logre
programar sus múltiples núcleos heterogéneos de manera automática, generando código paralelo a 
partir de código secuencial en C.



\section{Síntesis del problema}

Inexistencia y falta de optimización de una herramienta de generación automática 
de código paralelo a partir de código secuencial C, para las plataformas MPSoC KeyStone de Texas Instruments.
