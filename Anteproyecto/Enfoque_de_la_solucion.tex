%% ---------------------------------------------------------------------------
%% Enfoque_de_la_solucion.tex
%%
%% Enfoque de la solución
%%
%% rojiark
%% ---------------------------------------------------------------------------


\chapter{Enfoque de la solución}
\label{ch:Enfoque_de_la_solucion}

El enfoque que se pretende dar a la solución se basa en la utilización de un lenguaje de programación
de alto nivel como lo es C, para el diseño e implementación de un algoritmo que permita la programación
de diferentes unidades de procesamiento en plataformas MPSoC. El desarrollo va a ser realizado en un ambiente
Linux que facilite el uso de los recursos de la plataforma así como la depuración y evaluación de los diferentes
componentes de hardware que se emplearan.\\

Además del lenguaje a utilizar, se pretende manejar otros elementos a nivel de software como lo son simuladores, 
depuradores y las bibliotecas para compilación cruzada, los cuales estarán integrados dentro de un ambiente de 
desarrollo (\textit{IDE, Integrated Development Enviroment}) de Texas Instruments que podrá ser descargado y 
utilizado sin algún costo. Sin embargo, deberá desarrollarse interfaces especificas 
(\textit{API's, Application Programming Interface}) que puedan integrarse al IDE de Texas Instruments para 
la correcta validación de lo realizado.\\

Agregado a lo anterior se manipularan plataformas de hardware destinadas a ser los elementos objetivo de la
programación, específicamente KeyStone I y II. La primera tiene como unidades de procesamiento 8 DSP's 
TMS320C66x, mientras que la segunda implementa en su arquitectura la misma cantidad y tipo de DSP's
mas 4 procesadores de propósito general ARM Cortex-A15.\\

También sera necesario utilizar diversos instrumentos de medición como analizadores lógicos y osciloscopios
que permitan la depuración mediante señales eléctricas de los comportamientos no deseados en el sistema en general.\\

Con lo anterior establecido entonces se podrá realizar la programación del algoritmo y su posterior depuración, para
determinar la eficiencia o calidad que va a tener el mismo en cuanto a consumo de recursos de los CPU/DSP (como registros,
memoria, unidades aritméticas, etc.) y contar con los datos suficientes y correctos que permitan establecer