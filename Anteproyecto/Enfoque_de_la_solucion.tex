\chapter{Enfoque de la solución}
\label{ch:Enfoque_de_la_solucion}

El enfoque que se pretende dar a la solución se basa en la utilizacion de un lenguage de programacion
de alto nivel como lo es C, para el diseno e implementacion de un algoritmo que permita la programacion
de diferentes unidades de procesamiento en plataformas MPSoC. El desarrollo va a ser realizado en un ambiente
Linux que facilite el uso de los recursos de la plataforma asi como la depuracion y evalacion de los diferentes
componentes de hardware que se emplearan.\\

Ademas del lenguaje a utilizar, se pretende manejar otros elementos a nivel de software como lo son simuladores, 
depuradores y las bibliotecas para compilacion cruzada, los cuales estaran integrados dentro de un ambiente de 
desarrollo (\textit{IDE, Integrated Development Enviroment}) de Texas Instruments que podra ser descargado y 
utilizado sin algun costo. Sin embargo, debera desarrollarse interfaces especificas 
(\textit{API's, Application Programming Interface}) que puedan integrarse al IDE de Texas Instruments para 
la correcta validacion de lo realizado.\\

Agregado a lo anterior se manipularan plataformas de hardware destinadas a ser los elementos objetivo de la
programacion, especificamente KeyStone I y II. La primera tiene como unidades de procesamiento 8 DSP's 
TMS320C66x, mientras que la segunda implementa en su arquitectura la misma cantidad y tipo de DSP's
mas 4 procesadores de proposito general ARM Cortex-A15.\\

Tambien sera necesario utilizar diversos instrumentos de medicion como analizadores logicos y osciloscopios
que permitan la depuracion mediante senales electricas de los comportamientos no deseados en el sistema en general.\\

Con lo anterior establecido entonces se podra realizar la programacion del algoritmo y su posterior depuracion, para
determinar la eficiencia o calidad que va a tener el mismo en cuanto a consumo de recursos de los CPU/DSP (como registros,
memoria, unidades aritmeticas, etc.) y contar con los datos suficientes y correctos que permitan establecer