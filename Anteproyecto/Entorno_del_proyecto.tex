%% ---------------------------------------------------------------------------
%% Entorno_del_proyecto.tex
%%
%% Entorno del proyecto
%%
%% rojiark
%% ---------------------------------------------------------------------------

\chapter{Entorno del proyecto}
\label{ch:Entorno_del_proyecto}

Los sistemas embebidos o empotrados se utilizan en muchos de los dispositivos
y equipos con los cuales el ser humano interacciona o tiene relación alguna en la actualidad. Desde un dispositivos
móviles como un teléfono celular, pasando por televisores de alta definición, equipos de seguridad
en medios de transporte y hasta en la industria aeroespacial, la cual requiere de los elementos mas robustos
y de soporte crítico, son ejemplos claros de la utilización de los sistemas embebidos.\\

En general, un sistema empotrado se considera como aquel conjunto de elementos de carácter electrónico y/o
computacional-informático que  se encuentran relacionados entre sí y que tienen como función una aplicación 
específica dada.\cite {LERTA} \cite{MK1} \\

Hasta hace algunos años la arquitectura de estos dispositivos generalmente fue la utilización de un único procesador, 
con la tendencia del escalado en frecuencia \cite{TAODES} para aumentar el rendimiento de los mismos. Sin embargo,
las tendencias han variado y han surgido los Sistemas en Chip de Múltiples Procesadores 
(\textit{MPSoC, Multi-Processor Systems on Chip}), los cuales integran una gran variedad de dispositivos entre los cuales
se encuentran elementos con múltiples núcleos de procesamiento.\\ 

Actualmente estos Sistemas en Chip (\textit{SoC, Systems-on-Chip}) incorporan 
múltiples núcleos de diferente naturaleza entre los cuales se encuentran los Procesadores de Propósito 
General (\textit{GPP, General Purpose Processors}) como la gama de procesadores ARM, Procesadores Digitales de 
Señales (\textit{DSP, Digital Signal Processors}) que se encuentran de diversos tipos como los 
especializados para imágenes (\textit{DIP, Digital Image Processor}), así también como los dedicados al tratamiento de gráficos llamados 
GPU (\textit{GPU, Graphics Processing Units}) \cite{MK2}. Todos los anteriores encapsulados en un mismo circuito integrado o bien 
incorporados en el SoC que los alberga.\\

Éste cambio en el paradigma de la computación de pasar de sistemas de un solo núcleo de procesamiento a los MPSoC trae consigo retos 
que antes no se conocían, tanto para los proveedores como para los desarrolladores de las plataformas. Los modelos de programación cambian 
de un estándar secuencial a uno paralelo que justamente aproveche el paralelismo inherente a los múltiples procesadores,
lo que conlleva nuevos métodos de programación, compilación, y ejecución que deben de cumplir y satisfacer los requerimientos 
característicos de los sistemas embebidos como operación en tiempo real \cite{LERTA_RT}, eficiencia energética y procesamiento limitado.\\

La Universidad de Aquisgrán (\textit{RWTH Aachen University}) es un centro académico y de investigación en diversas
ramas de la ingeniería. En la actualidad es reconocida mundialmente por los trabajos realizados en las áreas de ingeniería mecánica, 
electrónica y ciencias de la computación \cite{RWTH_AACHEN_REF}. Dentro de su sistema académico, la universidad cuenta con diversos centros
de investigación (institutos) que realizan aportes de vanguardia con nuevas tecnologías y avances, concretando éstos mediante publicaciones en revistas
internacionales de alto renombre. Estos centros de investigación son dirigidos mediante Juntas (\textit{Chairs}) 
que tienen a cargo diversos proyectos tanto en áreas especificas como en áreas multidisciplinarias.\\

Uno de los institutos con mas publicaciones y aportes a la ciencia es el Instituto de Tecnologías de Comunicaciones y 
Sistemas Embebidos (\textit{ICE, Institute for Communications Technologies and Embedded Systems}) \cite{ICE_ABOUT}, 
el cual integra tres de las juntas de mayor importancia en el área de la electrónica y las ciencias 
de la computación (\textit{SSS, Software for Systems on Silicon; ISS, Integrated Signal Processing; Research Group 
MPSoC Architectures UMIC Research Cluster}); estos "\textit{Chairs}" colaboran entre si en la generación de conocimiento, y cuentan con 
tres áreas vitales de las comunicaciones eléctricas y los sistemas embebidos. 

\begin{itemize}
 \item Algoritmos y esquemas de transmisión y recepción en los sistemas de comunicaciones inalámbricas.
 \item Hardware y arquitecturas de los chips para sistemas embebidos, en particular para los MPSoC.
 \item Diseño de herramientas a nivel de sistema (\textit{ESL, Electronic System-Level}) para sistemas y circuitos integrados complejos.
\end{itemize}

En la actualidad el Instituto cuenta con una aplicación llamada MAPS (\textit{MAPS, MPSoC Application Programming Studio}) \cite{MAPS1}, la cual
consiste en un sistema de herramientas que asisten a los programadores en la paralelización de código secuencial C. La interfaz del usuario
esta basada en las herramientas de C/C++ del ambiente de desarrollo Eclipse.\\

A pesar de la gran cantidad de tiempo que se ha invertido en el desarrollo de la aplicación MAPS, ésta no se encuentra concluida y solo 
cuenta con soporte para determinadas plataformas de desarrollo, entre ellas un soporte limitado para la KeyStone I \cite{KSI1}, la cual es
una plataforma de múltiples unidades de procesamiento (\textit{PE, Processing Elements}) homogéneas, en este caso DSP's. Parte del soporte 
que del todo no se encuentra es para las arquitecturas de KeyStone II \cite{KSII1}, cuya estructura consta de unidades de procesamiento 
pero heterogéneas, es decir, procesadores de propósito general ARM mas DSP's.


%%% Local Variables: 
%%% mode: latex
%%% TeX-master: "main"
%%% End: 